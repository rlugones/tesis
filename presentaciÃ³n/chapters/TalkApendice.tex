\frame{\frametitle{{\Large Apéndice}}}

\frame{\frametitle{Apéndice: Espectro de energías axisimétrico}
\begin{itemize}
  \item Espectro de energías axisimétrico $e(k_\perp, k_\parallel, t)$\\
    \begin{equation*}
      \begin{split} e(k_\perp, k_\parallel, t) = \sum_{\substack{k_\perp \leq |\vec{k}\times\hat{x}| < k_\perp+1 \\ k_\parallel \leq k_x < k_\parallel +1}} |\hat{u}(\vec{k},t)|^2 +|\hat{b}(\vec{k},t)|^2 = \\ = \int \left(|\hat{u}(\vec{k},t)|^2 +|\hat{b}(\vec{k},t)|^2\right) |\vec{k}| \sin \theta_k~d\phi_k
      \end{split}
    \end{equation*}
  \item Espectros energéticos perpendiculares reducidos $E(k_\perp)$
    \begin{equation*}
      E\left(k_\perp\right) = \frac{1}{T}\int\int e(|\vec{k_\perp}|,
      k_\parallel, t) \, dk_\parallel~dt,
    \end{equation*}
\end{itemize}
}

\frame{\frametitle{Apéndice: Demostración contrapropagación de $\vec{z}^-$}
Ecuación ideal linealizada ($\rho$ constante, $\vec{U}=0$)
  \begin{equation*}
    \partial_t \vec{z}^\pm = \pm \vec{V}_{A} \cdot \nabla \vec{z}^\pm 
    \mp \vec{z}^\mp \cdot \frac{\nabla \vec{B}'}{\sqrt{4\pi \rho}} ,
  \end{equation*}
donde $\vec{V}_{A}$ incluye variaciones de gran escala y $\vec{B}'$ es el
campo magnético total en unidades gaussianas. \vfill

Si $\sigma_c \sim 1$ (i.e., $|\vec{z}^+| \gg |\vec{z}^-|$), tenemos
que para $\vec{z}^+$
\begin{equation*}
\partial_t \vec{z}^+  \approx \vec{V}_{A} \cdot \nabla \vec{z}^+ ,
\end{equation*}
Utilizando $\vec{z}^\pm = \vec{z}_0^\pm e^{i(\vec{k}\cdot
  \vec{x}+\omega^\pm t)}$ recuperamos $\omega^+ =
+\vec{V}_{A} \cdot \vec{k}$.
}

\frame{\frametitle{Apéndice: Demostración contrapropagación de $\vec{z}^-$}
No obstante, para $\vec{z}^-$,
\begin{equation*}
\partial_t \vec{z}^- \approx - \vec{V}_{A} \cdot \nabla \vec{z}^- + \vec{z}^+ 
    \cdot \frac{\nabla \vec{B}'}{\sqrt{4\pi \rho}} .
\end{equation*}

Utilizando $\vec{z}^\pm = \vec{z}_0^\pm
e^{i(\vec{k}\cdot \vec{x}+\omega^\pm t)}$, y asumiendo $\vec{B}' =
\vec{B}_0 + \vec{b}_0'$ donde $\vec{b}_0' =
\vec{\tilde{b}}_0'e^{i\vec{K} \cdot \vec{x}}$ es el campo magnético de
gran escala lentamente variable, con número de onda $K \ll k$, la ecuación se reduce a
\begin{equation*}
\left( \omega^- +\vec{V}_{A} \cdot \vec{k} \right) 
    \vec{z}_0^- e^{i \omega^- t} = 
    \frac{\left(\vec{K} \cdot \vec{z}_0^+\right) \vec{b}_0'}{\sqrt{4\pi \rho}} 
    e^{i \omega^+ t} .
\end{equation*}
Tomando el producto escalar con $\vec{z}_0^-$, definiendo las
densidades energéticas de Els\"asser $e^\pm = |\vec{z}_0^\pm|^2/4$, y
definiendo las fluctuaciones en la velocidad de Alfvén (asociadas a
las fluctuaciones de gran escala del campo magnético) como
$\vec{v}_{A} = \vec{b}_0'/\sqrt{4\pi \rho}$, obtenemos
\begin{equation*}
\left( \omega^- + \vec{V}_{A} \cdot \vec{k} \right)
    e^{i \omega^- t} = 
    \frac{\left(\vec{K} \cdot \vec{z}_0^+\right)
    \left(\vec{v}_{A} \cdot \vec{z}_0^-\right)}
    {4e^-} 
    e^{i \omega^+ t} .
\end{equation*}
}
\note[itemize]{
\item $\vec{z}^-$ (más pequeñas $\vec{z}^+$) pueden
ser afectadas fuertemente por el campo $\vec{z}^+$ y por las
variaciones espaciales del campo magnético de gran escala.

}


\frame{\frametitle{Apéndice: Demostración contrapropagación de $\vec{z}^-$}
Esta ecuación admite como soluciones
\begin{eqnarray*}
    \omega^- &=& \omega^+ =
    + \vec{V}_{A} \cdot \vec{k}, 
    \label{eq4:cond1} \\
    2 \vec{V}_{A} \cdot \vec{k} &=& 
    \left(\vec{K} \cdot \vec{z}_0^+\right)
    \left(\vec{v}_{A} \cdot \vec{z}_0^-\right) /
    (4e^-), \label{eq4:cond2}
\end{eqnarray*}

A partir de análisis dimensional, la 2da condición requiere
\begin{equation*}
    2 \frac{V_{A}}{v_{A}} \frac{k}{K}
    \sim \sqrt{\frac{e^+}{e^-}},
\end{equation*}
que (como $V_{A}\gtrsim v_{A}$ y $k\gg K$) no puede ser
satisfecha cuando $\sigma_c \approx 0$ (como se observa en los espectros),
o cuando el campo guía se vuelve demasiado
fuerte para un valor fijo de $\sigma_c$ (como también fue observado en
los espectros espacio-temporales).\vfill

Entonces, las fluctuaciones $\vec{z}^-$ pueden propagarse con la misma velocidad de
fase y dirección que las fluctuaciones $\vec{z}^+$, siempre que
$\sigma_c \neq 0$ y $B_0$ no sea demasiado fuerte para un valor fijado
de $\sigma_c$.
}
\note[itemize]{
\item ambas ondas viajando en la misma dirección mientras
la segunda condición se cumpla
}
