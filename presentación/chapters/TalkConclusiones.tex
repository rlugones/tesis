\frame{\frametitle{Conclusiones generales}
  \begin{itemize}
  \item Estudiamos el tiempo de
descorrelación (en función de $\vec{k}$) y los espectros
espacio-temporales (en función de $\omega$ y $\vec{k}$) para distintos $B_0$ y $\sigma_c=0$.
\item Los efectos no locales juegan un rol preponderante en la
turbulencia MHD, y que las descorrelaciones están principalmente
dominadas por el \textit{sweeping} y por las interacciones Alfvénicas 
\item Conseguimos distinguir entre los dos efectos no locales.
\item Las ondas se encuentran todavía presentes en
turbulencia MHD y dominan las descorrelaciones esencialmente para
números de onda paralelos al campo medio.
\item Para $\sigma_c=0$ el sistema elige el tiempo de descorrelación más bajo
disponible.

  \end{itemize}
}
\note[itemize]{
\item 2) Confirmando los resultados previos obtenidos para el caso isotrópico
\item 3) El \textit{sweeping} domina la descorrelación para
valores moderados de $B_0$, y Alfvén para $B_0$ grande y a
grandes escalas (aunque aun en este caso, para escalas pequeñas
el \textit{sweeping} vuelve a ser el efecto dominante).
}

\frame{\frametitle{Conclusiones generales}
  \begin{itemize}
  \item Al variar $\sigma_c$, nos encontramos con nuevos resultados.
  \item Para valores grandes de $B_0$ o de $\sigma_c$, los tiempos
  de descorrelación están controlados por los efectos Alfvénicos,
  aún cuando $\tau_A$ no sea el más pequeño.
  \item Este efecto es posiblemente causada por una
transición hacia un régimen con no-linealidades más débiles a medida
que se incrementa la helicidad cruzada.
\item También encontramos un régimen en el que fluctuaciones con
polarizaciones opuestas se propagan en la misma dirección debido a la
reflexión de ondas, causada por inhomogeneidades del campo magnético
de gran escala.
\item Este fenómeno se vuelve más prominente en flujos con alta $\sigma_c$, modificando
fuertemente la propagación de ondas en flujos MHD turbulentos y las
interacciones no lineales en el medio interplanetario.
  \end{itemize}
}
\note[itemize]{
\item 4) Este resultado confirma las
conclusiones de Hollweg y provee evidencia de un
fenómeno predicho anteriormente con la teoría WKB.
\item En MHD, la propagación de la onda de Alfvén como
el \textit{sweeping} contribuyen a la variación de tiempo total en un
punto (espectro de frecuencia euleriano) y, por lo tanto, influyen en
una predicción limitante.
\item Todos estos resultados muestran la complejidad y la multiplicidad de
fenómenos presentes en el sistema de MHD, aún en el caso
incompresible.
}


\frame{\frametitle{Posibles caminos, futuros y no tanto}
Este estudio podría extenderse a muchos otros casos, buscando
una mayor comprensión de la física del plasma turbulento.
  \begin{itemize}
  \item MHD compresible
  \item Helicidad cinética
  \item Helicidad magnética
  \item Helicidad híbrida
  \item etc...
  \end{itemize}

También podrían estudiarse los efectos de otras condiciones de
contorno (no periódicas).
}
\note[itemize]{
\item Nahuel
\item Mauro
}
