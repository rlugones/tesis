A lo largo de esta Tesis, hemos estudiado el comportamiento
espacio-temporal de las fluctuaciones de diversos campos presentes en
el sistema: primero el campo magnético (\cref{ch:P1}), y segundo los
campos de Elss\"asser (\cref{ch:P2}).

En primer lugar, en el \cref{ch:P1}, estudiamos el tiempo de
descorrelación (en función de $\vec{k}$) y los espectros
espacio-temporales (en función de $\omega$ y $\vec{k}$) para distintos
valores del campo magnético medio de fondo, en el caso de helicidad
cruzada nula. Los resultados encontrados respaldan la conclusión de
que los efectos no locales juegan un rol preponderante en la
turbulencia MHD, y que las descorrelaciones están principalmente
dominadas por el \textit{sweeping} y por las interacciones Alfvénicas,
confirmando los resultados previos obtenidos para el caso isotrópico
[\cite{servidio_time_2011}]. Adicionalmente, el análisis realizado
permite distinguir entre los dos efectos no locales, y los resultados
apoyan la conclusión de que la interacción de \textit{sweeping} domina
la descorrelación para valores moderados de $B_0$, mientras que para
grandes valores del campo medio $B_0$ y a grandes escalas las
descorrelaciones están más controladas por las interacciones
Alfvénicas (aunque aun en este caso, para escalas pequeñas
el \textit{sweeping} vuelve a ser el efecto dominante). Las
interacciones relevantes son las ondas de Alfvén, y como tales se
puede concluir que las ondas se encuentran todavía presentes en
turbulencia MHD y dominan las descorrelaciones esencialmente para
números de onda paralelos al campo medio. En estos casos estudiados,
se observa que el sistema elige el tiempo de descorrelación más bajo
disponible.

Posteriormente, en el \cref{ch:P2}, estudiamos el caso de los campos
de Elss\"asser correspondientes a las fluctuaciones de los campos
magnético y de velocidades. Los resultados indicaron que
el \textit{sweeping} domina los tiempos de descorrelación para valores
bajos del campo magnético medio y de la helicidad cruzada; mientras que
para valores grandes del campo magnético medio o de helicidad
cruzada, los tiempos de descorrelación están controlados por los
efectos Alfvénicos. Esto sucede aún cuando el tiempo de Alfvén no sea
el más rápido, lo que resulta en una nueva característica respecto de
estudios previos del comportamiento espacio-temporal de turbulencia
MHD fuerte con helicidad cruzada nula, posiblemente causada por una
transición hacia un régimen con no-linealidades más débiles a medida
que se incrementa la helicidad cruzada.

También encontramos un régimen en el que fluctuaciones con
polarizaciones opuestas se propagan en la misma dirección debido a la
reflexión de ondas, causada por inhomogeneidades del campo magnético
de gran escala. De esta forma, este resultado confirma las
conclusiones de \cite{hollweg_1990_wkb} y provee evidencia de un
fenómeno predicho anteriormente con la teoría WKB. Este fenómeno se
vuelve más prominente en flujos con alta helicidad cruzada,
modificando fuertemente la propagación de ondas en flujos
magnetohidrodinámicos turbulentos y las interacciones no lineales en
el medio interplanetario.

\vspace{0.7cm}

En MHD, tanto la propagación de la onda de Alfvén como
el \textit{sweeping} contribuyen a la variación de tiempo total en un
punto (espectro de frecuencia euleriano) y, por lo tanto, influyen en
una predicción limitante. De esta forma, si bien no resulta sencillo
extrapolar las conclusiones de forma precisa a aplicaciones espaciales
y astrofísicas, sí permiten realizar un análisis cualitativo y pueden
ayudar a explicar ciertos resultados experimentales, como hemos
expuesto en las secciones \ref{sec3:Conclusions}
y \ref{sec4:Conclusions}.

Todos estos resultados muestran la complejidad y la multiplicidad de
fenómenos presentes en el sistema de MHD, aún en el caso
incompresible. Sin dudas, resulta necesario un análisis más detallado
para poder comprender y analizar toda la física presente en plasma
turbulento.  En futuros trabajos, por ejemplo, podría extenderse este
estudio a MHD compresible [\cite{andres_2017_interplay}], considerando
la dependencia con la helicidad cruzada en el flujo y su interacción
con los efectos compresibles. También podría realizarse un estudio que
considere otras helicidades, como la helicidad cinética $H_v$, la
helicidad magnética $H_b$ y la helicidad híbrida para
Hall-MHD. Asimismo, se podría analizar los efectos de distintos tipos
de forzados compresibles e incompresibles, así como también estudiar
qué sucede variando el número de Mach.

Cualquiera de estos estudios sería un seguimiento interesante de la
presente Tesis y un primer paso hacia una comprensión más profunda
del papel de los efectos no lineales en la propagación de ondas en la
turbulencia en plasma. 



%%%%%%%%%%%%%%%%%%%%%%%%%%%%%%%%



%% \newpage


%% {\color{olive}

%% \section{P1}
%% Utilizando simulaciones numéricas de turbulencia MHD 3D,
%% analizamos el comportamiento espacio-temporal de las fluctuaciones del
%% campo magnético. Consideramos casos con un campo magnético medio de
%% fondo fuerte, mediano y débil.  Computamos la función de correlación
%% temporal dependiente de la escala espacial (vía el número de onda)
%% para las distintas simulaciones realizadas, variando el campo
%% magnético medio.  A partir de estas funciones de correlación,
%% computamos el tiempo de descorrelación y lo comparamos con diferentes
%% tiempos teóricos, a saber, el tiempo local no lineal, el tiempo
%% de \textit{sweeping} aleatorio y el tiempo de Alfvén, siendo este
%% último un efecto ondulatorio.  Observamos que los tiempos de
%% descorrelación están dominados por los efectos de \textit{sweeping}, y
%% sólo para valores grandes del campo magnético medio y para vectores de
%% onda mayormente alinealos con este campo, los tiempos de
%% descorrelación están controlados por efectos Alfvénicos.

%% \rule{4cm}{0.4pt}

%% En este trabajo, hemos estudiado los tiempos de correlación que entran
%% en juego en magnetohidrodinámica, en la aproximación
%% incompresible. Aún en el caso (más simple) hidrodinámico, uno espera
%% que tanto las correlaciones espaciales como las temporales sean
%% relevantes en la física de la turbulencia, ya que estas propiedades
%% independientes puede encarnarse en el tensor de correlación de dos
%% puntos y dos tiempos, $R_{ij}(\vec{r},t)$, una generalización directa
%% de la \cref{eq3:Rbij}. Correlaciones análogas pueden ser escritas para
%% las componentes de la velocidad del fluido $\vec{v}$ y para otras
%% cantidades. La transformada espacial de la correlación (o,
%% equivalentemente, las funciones de estructura espacial de segundo
%% orden) a tiempo de retraso $\tau$ nulo, proveen información acerca de
%% la distribución espacial de la energía a lo largo de las distintas
%% escalas. Acordemente, la correlación temporal a un punto espacial, variando
%% el tiempo de retraso y transformándolo en frecuencias, provee
%% información análoga acerca de la distribución energética a lo largo de
%% las distintas escalas temporales. Aquí, estudiamos las correlaciones
%% en tiempo para un dado número de onda o una escala espacial para el
%% modelo magnetohidrodinámico.

%% El caso MHD es más complejo que el hidrodinámico porque hay dos campos
%% involucrados: el magnético y el de velocidades. Además, el campo
%% magnético no puede ser removido por una transformada de Galileo,
%% mientras que el de velocidades, sí. En consecuencia, el campo
%% magnético medio, impone una dirección preferencial. Adicionalmente, el
%% caso MHD tiene un nuevo y anisotrópico modo de ondas, las ondas de
%% Alfvén, que introducen la posibilidad de anisotropías en el espectro y
%% en las correlaciones, así como también una nueva escala temporal, el
%% tiempo de Alfvén. Debido a estos efectos, el análisis de la
%% descorrelación temporal se vuelve también más complejo, con al menos
%% tres escalas temporales para examinar (Alfvén, \textit{sweeping} y no lineal),
%% así como también la posibilidad de una anisotropía en la tasa de
%% descorrelación.

%% Tanto el \textit{sweeping} aleatorio como la correlación Alfvénica son efectos
%% no locales, en el sentido de que acoplan las grandes escalas con otras
%% más pequeñas. Los resultados mostrados aquí respaldan la conclusión de
%% que los efectos no locales (en el espacio espectral) juegan un rol
%% importante en turbulencia MHD (en acuerdo con los estudios de
%% transferencia \textit{shell-to-shell} introducidos en el \cref{ch:fundamentos}),
%% y que las descorrelaciones están principalmente dominadas por el
%% \textit{sweeping} y las interacciones Alfvénicas, confirmado los estudios
%% previos de MHD isotrópico [\cite{servidio_time_2011}].

%% Además, en comparación con los estudios previos, el análisis aquí
%% presentado permite distinguir entre los efectos de \textit{sweeping} y
%% Alfvénicos, y los resultados apoyan la conclusión de que la
%% interacción de \textit{sweeping} domina la descorrelación para valores
%% moderados de $B_0$, mientras que para grandes valores del campo medio
%% $B_0$ y a grandes escalas (números de onda perpendiculares pequeños)
%% las descorrelaciones están más controladas por las interacciones
%% Alfvénicas.  Las interacciones relevantes son las ondas de Alfvén, y
%% como tales se puede concluir que las ondas se encuentran todavía
%% presentes en turbulencia MHD y dominan las descorrelaciones
%% esencialmente para números de onda paralelos (alineados con el campo
%% medio; ver también \cite{meyrand_direct_2016,
%% meyrand_weak_2015}). Nuestros resultados también indican que el
%% sistema elige, en efecto, el tiempo de descorrelación más bajo
%% disponible. Un constructo simple y relevante es que la tasa de
%% descorrelación es la suman de las tasas asociadas con cada escala de
%% tiempo relevante (ver, por ejemplo, \cite{pouquet_strong_1976,
%% zhou_magnetohydrodynamic_2004}). Como resultado, aún para grandes
%% valores del campo guía $B_0$, para escalas suficientemente pequeñas en
%% las que el tiempo de \textit{sweeping} resulte más rápido que el de
%% Alfvén, luego de un gran rango de escalas en las que dominen las ondas
%% de Alfvén, el sistema transiciona a un comportamiento donde domina el
%% \textit{sweeping}.


%% Una conclusión convincente del presente trabajo es que la influencia
%% de la descorrelación de \textit{sweeping} se extiende a lo largo de un amplio
%% rango de los parámetros globales. Aún si el \textit{sweeping} no es el tiempo
%% dominante de los mecanismos de descorrelación a lo largo de todo el
%% sistema, su importancia relativa a la descorrelación vía propagación
%% Alfvénica persiste en ciertas subregiones del espacio de Fourier.
%% Este es el caso para valores moderados del campo magnético medio
%% aplicado $B_0$, como puede observarse en las
%% \cref{fig3-5:B1_bvf_b_kperp,fig3-5:B1_bvf_b_kpara}. Esta influencia del
%% \textit{sweeping} se encuentra aún en los casos con campo magnético medio
%% fuerte ($B_0 = 8$), como se ve en
%% las \cref{fig3-5:B8_bvf_b_kperp,fig3-5:B8_bvf_b_kpara}. Acordemente,
%% también se podría concluir que los efecto de descorrelación Alfvénica
%% son muy importantes, por lo menos para valores altos de $B_0$ y en
%% ciertas regiones del espacio de ondas.  A pesar de que resulta difícil
%% extrapolar tales conclusiones en una forma precisa para aplicaciones
%% espaciales y astrofísicas, podemos aplicar los presentes resultados en
%% una forma cualitativa.  Por ejemplo, el viento solar típicamente
%% admite $\delta B/B_0 \sim 1$ en la escala más externa.  Aún si el
%% cociente es menor, por ejemplo a escalas menores en el rango inercial,
%% el presente resultado sugiere que el efecto de \textit{sweeping} se
%% mantendría importante en establecer la tasa del tiempo de
%% descorrelación en el ambiente interplanetario. Esto podría conllevar
%% diversas implicaciones, por ejemplo en la predicción cuantitativa, en
%% la dispersión de partículas y en la comprensión del ámbito de
%% aplicación de la teoría de la turbulencia débil. En este sentido, las
%% técnicas de observación han comenzado a extraer medidas aproximadas
%% del viento solar y la descorrelación del tiempo magnetosférico en el
%% marco del plasma
%% [\cite{matthaeus_ensemble_2016,weygand_magnetic_2013}], pero aún no han
%% alcanzado la precisión para distinguir los efectos de barrido y
%% Alfvénicos como lo ha hecho el presente estudio utilizando la
%% simulación MHD.

%% Es interesante recordar que la descorrelación de tiempo relevante
%% asociada con la transferencia de energía en turbulencia no es la
%% correlación de tiempo euleriana que hemos considerado (punto espacial
%% fijo, tiempo variable), sino más bien la descorrelación de tiempo
%% lagrangiana, calculada siguiendo un elemento fluido material. A este
%% respecto, es bien sabido que ni el barrido ni la propagación de ondas
%% Alfvénicas pueden producir directamente la transferencia espectral en
%% modelos homogéneos idealizados. En parte debido a estas
%% complicaciones, actualmente no existe una teoría completa que vincule
%% la correlación espacial y las correlaciones de tiempo en MHD o
%% turbulencia hidrodinámica. Por otro lado, está claro que en MHD, tanto
%% la propagación de la onda de Alfvén como el \textit{sweeping} contribuyen a la
%% variación de tiempo total en un punto (espectro de frecuencia
%% euleriano) y, por lo tanto, influyen en una predicción
%% limitante. Estas escalas de tiempo también son características
%% importantes para comprender la dispersión de partículas de prueba
%% cargadas, como los rayos cósmicos de baja energía
%% [\cite{bieber_proton_1994}], así como para tener en cuenta la
%% distribución de las aceleraciones, que está relacionada con la
%% intermitencia [\cite{nelkin_time_1990}].

%% El comportamiento observado del tiempo de descorrelación para MHD,
%% ejemplificado por los nuevos resultados presentados aquí, tiene
%% aplicaciones en una serie de temas, incluyendo la teoría de dispersión
%% de partículas cargadas [\cite{schlickeiser_cosmic-ray_1993,
%%   nelkin_time_1990}], la dinámica del campo magnético interplanetario y
%% de la magnetosfera [\cite{miller_critical_1997}], y la interpretación de
%% datos de naves espaciales de misiones históricas y futuras
%% [\cite{matthaeus_ensemble_2016}]. Mirando hacia las perspectivas
%% futuras, notamos que ha habido cierto éxito en el establecimiento de
%% conexiones empíricas entre la escala de tiempo de \textit{sweeping} y la
%% descorrelación del tiempo euleriano observado en hidrodinámica
%% [\cite{chen_sweeping_1989}]. Se podrían aprovechar ideas similares para
%% MHD (por ejemplo, \cite{matthaeus_dynamical_1999}) para comprender
%% mejor, o al menos modelar empíricamente, la relación en MHD entre la
%% estructura espacial y la descorrelación del tiempo, un esfuerzo que se
%% beneficiaría directamente de los resultados novedosos presentados
%% aquí.







%% \newpage
%% \section{P2}
%% Estudiamos el comportamiento espacio-temporal de las variables
%% de Els\"asser correspondientes a las fluctuaciones de los campos
%% magnético y de velocidades, utilizando simulaciones numéricas de MHD
%% 3D turbulento. Consideramos casos con campo magnético medio de fondo
%% débil, mediano y fuerte, y con helicidad cruzada (correlaciones entre
%% el campo de velocidades y el magnético) nula, pequeña y grande.
%% Computamos las funciones de correlación temporal para cada una de las
%% simulaciones.  A partir de dichas funciones de correlación, computamos
%% los tiempos de descorrelación y los comparamos con los distintos
%% tiempos teóricos característicos: tiempo local no lineal, tiempo
%% de \textit{sweeping} aleatorio y tiempo de Alfvén. Encontramos que los
%% tiempos de descorrelación son dominados por los efectos
%% de \textit{sweeping} para valores bajos del campo magnético medio y
%% para valores bajos de la helicidad cruzada; mientras que para valores
%% grandes del campo de fondo o de la helicidad cruzada, y para valores
%% del vector de onda suficientemente alineados con el campo guía,los
%% tiempos de descorrelación son controlados por efectos
%% Alfvénicos. Finalmente, observamos contrapropagación de fluctuaciones
%% Alfvénicas debidas a reflexiones producidas por inhomogeneidades en el
%% campo magnético total. Este efecto se vuelve más prominente el flujos
%% con alta helicidad cruzada, modificando fuertemente la propagación de
%% ondas en flujos magnetohidrodinámicos turbulentos.

%% \rule{4cm}{0.4pt}

%% Analizamos el comportamiento espacio-temporal de las fluctuaciones MHD
%% considerando sus polarizaciones en términos de las variables de
%% Els\"asser, utilizando simulaciones numéricas directas
%% tridimensionales de turbulencia MHD incompresible. Consideramos casos
%% con valores bajos, intermedios y altos de campo magnético medio de
%% fondo, y con helicidad cruzada nula, pequeña y alta. Las funciones de
%% correlación como función de los vectores de onda (descompuestos en las
%% direcciones perpendicular y paralela del campo magnético medio) y del
%% tiempo de retraso fue directamente computada para todas las diferentes
%% simulaciones consideradas, así como también los espectros
%% espacio-temporales. A partir de las funciones de correlación,
%% computamos el tiempo de descorrelación para cada modo de Fourier, y lo
%% comparamos con las diferentes predicciones teóricas para las escalas
%% temporales relevantes en el sistema: el tiempo no lineal local, el
%% \textit{sweeping} aleatorio y el tiempo Alfvénico. Se observó que los tiempos
%% de descorrelación son dominados por los efectos de \textit{sweeping} para
%% valores bajos del campo magnético medio y de la helicidad cruzada,
%% mientras que para valores grandes del campo magnético medio o de
%% helicidad cruzada, los tiempos de descorrelación están controlados por
%% los efectos Alfvénicos aún cuando el tiempo de Alfvén no sea el más
%% rápido, una nueva característica cuando lo comparamos con estudios
%% previos del comportamiento espacio-temporal de turbulencia MHD fuerte
%% con helicidad cruzada nula. En principio, este comportamiento puede
%% ser interpretado como una transición hacia un régimen con
%% no-linealidades más débiles a medida que se incrementa la helicidad
%% cruzada, como suele ser discutido teóricamente y como aparentemente
%% indican nuestras simulaciones numéricas.

%% Sin embargo, debe notarse que los espectros espacio-temporales indican
%% que aún en este régimen, las interacciones no lineales son
%% relevantes. El otro resultado principal obtenido de nuestro análisis
%% es el haber encontrado un régimen en el que se generan fluctuaciones
%% $\vec{z}^-$ y $\vec{z}^+$ (polarizaciones opuestas), y se propagan en
%% la misma dirección debido a la reflexión de ondas, causada por
%% inhomogeneidades del campo magnético de gran escala. Este resultado es más
%% evidente en los espectros espacio-temporales de los campos de
%% Els\"asser para valores intermedios del campo magnético de fondo (es
%% decir, cuando la componente uniforme y constante del campo magnético
%% de gran escala no es demasiado fuerte). Un análisis fenomenológico
%% basado en ideas previas en \cite{zhou1990remarks}
%% confirma las conclusiones de \cite{hollweg_1990_wkb}, que indican que
%% las fluctuaciones Alfvénicas con polarizaciones opuestas pueden
%% efectivamente propagarse en la misma dirección y aún con la misma
%% velocidad.  Si el campo magnético de fondo se vuelve demasiado fuerte
%% (o si la helicidad cruzada se acerca a cero), este efecto no se
%% observa más. Así, el análisis espacio-temporal de los flujos
%% turbulentos provee evidencia directa de un fenómeno predicho
%% anteriormente con la teoría WKB, y puede jugar un rol relevante
%% modificando la propagación de ondas y las interacciones no lineales en
%% el medio interplanetario.

%% Los resultados analizados en este trabajo muestran en detalle que, al
%% menos en el régimen de turbulencia fuerte, la representación de ondas
%% no es suficientemente completa para describir el sistema de MHD
%% incompresible. En este sistema aparece una amplia banda de
%% fluctuaciones provenientes de efectos locales y no locales
%% (\textit{sweeping}), que generan dispersión y efectos no lineales. Es
%% importante recordar, por supuesto, que gran parte del presente estudio
%% se ha concentrado en el estudio del tiempo de descorrelación euleriano,
%% descompuesto en un tiempo de descorrelación dependiente de la escala de
%% los modos de Fourier individuales. Esta descorrelación generalmente se
%% interpreta como una competencia entre la descorrelación por \textit{sweeping}
%% por fluctuaciones a gran escala y la descorrelación que se origina en
%% la propagación de ondas. Sin embargo, ninguno de estos efectos es en
%% principio responsable de la transferencia espectral que da lugar a la
%% cascada de turbulencias. De hecho, el efecto principal de la
%% propagación de Alfvén, desde la perspectiva de la cascada energética
%% en turbulenta fuerte, no es causar transferencia espectral sino
%% suprimirla [\cite{shebalin_1983_anisotropy}]. La comprensión de la
%% cascada en sí misma requiere examinar la fuerza de las no
%% linealidades. En este caso, el tiempo característico apropiado se
%% convierte en el tiempo no lineal, cuyo aislamiento requiere el
%% análisis de escalas de tiempo en el marco lagrangiano
%% [\cite{kraichnan_1964_kolmogorov}] (tener en cuenta que sólo en algunos
%% casos particulares en nuestro análisis, el tiempo no lineal se
%% identificó positivamente como un candidato para el tiempo de
%% descorrelación). Sin embargo, hemos demostrado que fenómenos
%% físicamente relevantes como la reflexión y la ``propagación anómala''
%% de fluctuaciones reflejadas pueden producir efectos observables en la
%% energía del flujo, y estos fenómenos se han reconocido en una variedad
%% de configuraciones de los diferentes parámetros de control del
%% sistema, con posibles aplicaciones.

%% Por ejemplo, efectos interesantes asociados con la reflexión se suman
%% a la complejidad de la dinámica, incluso en el caso más simple de MHD
%% incompresible considerado aquí. Esto tiene implicaciones importantes
%% para aplicaciones como el calentamiento coronal, la aceleración del
%% viento solar y la energización de partículas en el espacio
%% interplanetario [\cite{velli_1993_propagation, matthaeus_1999_coronal}].
%% Como otro ejemplo, las fluctuaciones observadas en el viento solar,
%% que tienden a alinear o antialinear el campo magnético y el campo de
%% velocidad (es decir, con diferentes polarizaciones Alfvénicas), no
%% siempre se pueden interpretar trivialmente como viajando
%% ``downstream'' o ``upstream'' respecto del campo magnético medio.




%% \rule{4cm}{2pt}
%% \newpage
%% Extensiones de este estudio a MHD compresible
%% [\cite{andres_2017_interplay}], considerando la dependencia con la
%% helicidad cruzada en el flujo y su interacción con los efectos
%% compresibles, así como un estudio que considera otras helicidades como
%% la helicidad cinética $H_v$, la helicidad magnética $H_b$ y la
%% helicidad híbrida para Hall-MHD, sería un seguimiento interesante del
%% presente estudio y un primer paso hacia una comprensión más profunda
%% del papel de los efectos no lineales en la propagación de ondas en la
%% turbulencia en plasma.

%% Para el capitulo 5 me parece bien la idea de poner las conclusiones
%% del paper 1, las del paper 2 y mencionar como perspectiva lo del
%% estudio del caso compresible, tal como esta, quizas agregando estudiar
%% distintos tipos de forzados compresibles/incompresibles y variando el
%% numero de Mach (que es algo que aun no hicimos).

%% }
