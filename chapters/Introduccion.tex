\section{Medio interplanetario}
El medio interplanetario llena el espacio entre planetas, cometas y
otros objetos del sistema solar. Lejos de ser vacío, este espacio está
lleno de plasma proveniente de la corona solar, radiación
electromagnética, rayos cósmicos, partículas de polvo y campos
magnéticos. El término \emph{viento solar} fue acuñado por Eugene
Parker \cite{Parker1958} para describir la expansión supersónica (y
supermagnetosónica) de la corona solar hacia el sistema solar. Esta
expansión se da debido a la diferencia de presiones entre la corona
solar y el espacio interplanetario que la rodea. La diferencia de
presiones genera la expansión del viento solar, a pesar de la
influencia gravitatoria de la estrella \cite{KivelsonYRussel1995}.

La dinámica de los plasmas espaciales, tales como el viento solar,
pueden mostrar grandes diferencias en sus propiedades físicas con
respecto a los plasmas presentes en el laboratorio. Por ejemplo, las
colisiones entre iones o electrones pueden ser bastante frecuentes en
el plasma coronal solar, pero son extremadamente raras en el medio
interplanetario a 1 unidad astronómica ($\si{1}{AU} =
\si{1.49e11}{m}$). Dado que algunas de estas condiciones pueden ser
imposibles de reproducir en laboratorios terrestres, el medio
interplanetario ofrece una oportunidad única para investigar una
amplia variedad de procesos plasmáticos. En particular, a
$\si{1}{AU}$, la densidad numérica media es de $\si{7}{cm^{-3}}$, la
velocidad de flujo (casi radial) es de $\si{450}{km/s}$, la
temperatura media es de $\si{1.2e5}{K}$ y el campo magnético medio
(valor absoluto) es de aproximadamente $\si{10}{nT}$ [valores
  extraídos de Kivelson y Russel, 1995].

El viento solar se modifica significativamente por la actividad del
Sol a través de cambios en el campo magnético solar, que terminan
contribuyendo al campo magnético interplanetario (IMF) y a la dinámica
general del plasma. Durante su expansión, el viento solar desarrolla
un régimen fuertemente turbulento, que puede estudiarse a través de
mediciones \emph{in situ} [Matthaeus y Goldstein, 1982]. La
turbulencia aparece como un estado de movimiento muy complejo y
fuertemente irregular en el espacio y el tiempo. Sin embargo, en un
momento dado, un flujo turbulento muestra la presencia de estructuras
organizadas de diferentes tamaños y diferentes tiempos característicos
que interactúan entre sí a medida que el flujo las convecta. Gracias a
las observaciones \emph{in situ} proporcionadas por diferentes naves
espaciales, el flujo del viento solar ofrece la mejor oportunidad para
estudiar directamente la dinámica no lineal en los plasmas espaciales
[p. Bruno y Carbone, 2013].

El plasma del viento solar está enhebrado con las líneas de campo
magnético interplanetario. Cuando el viento solar encuentra el campo
magnético intrínseco de la Tierra, en algunas condiciones particulares
del IMF, el proceso de reconexión magnética puede tener lugar [Dungey,
  1961]. La reconexión magnética implica un cambio de topología de un
conjunto de líneas de campo, mientras convierte la energía magnética
libre en calor y energía cinética. La reconexión magnética puede
ocurrir en una variedad de escenarios, desde la evolución de las
erupciones solares, las eyecciones de masa coronal hasta la formación
de estrellas [Biskamp, 2000, Priest y Forbes, 2000].

Cuando el plasma de viento solar supersónico y supermagnetosónico se
encuentra con un obstáculo, como una magnetosfera planetaria, se forma
un arco de choque. El arco de choque ralentiza el plasma del viento
solar entrante de velocidad supersónica a velocidad subsónica para que
la información sobre el obstáculo pueda propagarse aguas arriba dentro
de la región conmocionada y dejar que el plasma se ajuste y fluya a su
alrededor [Thomsen et al., 1983]. Aguas arriba del choque, sin
embargo, el viento solar no es consciente de que se está acercando a
un obstáculo. La región aguas arriba del choque y conectada
magnéticamente a ella, se conoce como \emph{foreshock}. En el choque
de proa, una pequeña fracción de las partículas del viento solar se
acelera y se propaga hacia la región de \emph{foreshock} aguas
arriba. Estas partículas de flujo posterior pueden conducir una serie
de inestabilidades de plasma, lo que lleva a la generación de ondas de
ultra baja frecuencia (ULF), que son detectadas por naves espaciales
[p. Tsurutani y Rodriguez, 1981].

Estos tres resultados de la dinámica del viento solar, es decir, su
régimen turbulento, el proceso de reconexión magnética y la generación
y distribución espacial de las ondas ULF, son el tema principal de la
presente Tesis.



\section{Plasma espacial}

Los plasmas espaciales son gases casi neutros compuestos
principalmente de protones y electrones (es decir, un plasma de
hidrógeno completamente ionizado) que están sujetos a fuerzas
eléctricas, magnéticas y probablemente a otros tipos de fuerzas. Al
mismo tiempo, estas cargas gratuitas actúan como fuentes de fuerzas
eléctricas y magnéticas. Debido a la naturaleza de largo alcance de
las fuerzas electromagnéticas, cada partícula cargada en el plasma
interactúa simultáneamente con una gran cantidad de otras partículas
cargadas. Este proceso da como resultado un comportamiento colectivo
del plasma.

Hay al menos tres niveles de descripción para modelar la dinámica de
los plasmas espaciales. La principal diferencia entre ellos son las
variables físicas utilizadas para describir el estado del plasma. Cuál
debe elegirse depende del tipo de fenómeno que le interese. Los tres
enfoques son:

(1) El movimiento de partículas cargadas individuales y su interacción
con el campo eléctrico y magnético.

(2) La descripción cinética de una colección de partículas.

(3) La descripción del fluido.

La forma más completa de especificar el estado de un plasma es dar las
posiciones y velocidades de todas las partículas y el valor de los
campos en cada punto del espacio. Para un sistema de $N$ partículas,
esta descripción de partículas implica un espacio de fase de $6N$
dimensiones, que se vuelve prohibitivamente grande a medida que
aumenta $N$ (por ejemplo, $N\sim 10^{15}$ para un cubo de
$\si{1}{km^3}$ en el medio interplanetario). Una posible salida a este
problema es utilizar la teoría cinética. En este enfoque estadístico,
definimos la función de distribución de velocidades $f_s(\vec{x},
\vec{v}, t)$, para cada especie $s$, de manera que $d^3x d^3v
f_s(\vec{x}, \vec{v}, t)$ es la cantidad de partículas en una caja de
tamaño $d^3x$ alrededor de $\vec{x}$, con velocidades en un cubo
$d^3v$ alrededor de $\vec{v}$, para un tiempo $t$. La función de
distribución $f_s$ satisface ecuaciones cinéticas tales como la
ecuación de Vlasov (en el límite no colisional). Estas ecuaciones se
acoplan con las ecuaciones de Maxwell, debido a la autoconsistencia
del campo electromagnético. A pesar de que es posible integrar este
conjunto de ecuaciones numéricamente, poder cubrir un rango realista
de valores de $(\vec{x}, \vec{v}, t)$ se vuelve muy demandante
computacionalmente. Un enfoque más simple comparativamente es
considerar una descripción fluídica del plasma, basada en los momentos
de menor orden de $f_s$ (tales como la densidad de partículas, la
velocidad media del flujo y la presión). Para considerar el hecho de
que las partículas cargadas interactúan con el campo electromagnético,
se acopla la ecuación del fluido con las ecuaciones de Maxwell. Dentro
de este marco, la aproximación más simple es el llamado modelo de
magnetohidrodinámica de un fluido (MHD).

La descripción MHD de un fluido describe adecuadamente la
fenomenología a grandes escalas temporales y espaciales.  Sin embargo,
a escalas espaciales y/o temporales más pequeñas, existen fenómenos
físicos que no pueden reproducirse con la descripción tradicional de
MHD de un fluido. Por ejemplo, para describir adecuadamente el
espectro de energía del viento solar derivado de observaciones
recientes [Sahraoui et al., 2009], o para estudiar la reconexión
magnética en el límite sin colisión, se requiere un marco teórico que
se extienda más allá de un fluido MHD.


\section{Viento solar turbulento}

La naturaleza produce habitualmente turbulencia MHD. Se puede
encontrar en varios entornos espaciales, como la corona solar, las
atmósferas planetarias o el medio interplanetario. Como discutimos
anteriormente, el viento solar se expande desde el Sol y penetra las
regiones entre los planetas. Una característica importante para
caracterizar un régimen turbulento estacionario e isotrópico de un
plasma es su espectro de potencia de energía $E_k$, que proporciona la
energía por unidad de número de onda. Al igual que en el caso
paradigmático de la turbulencia hidrodinámica, las interacciones no
lineales en la turbulencia MHD producen un flujo de energía en el
espacio de número de onda que es predominantemente de escalas grandes
a escalas pequeñas. Como se indica esquemáticamente en la Figura 1.2,
el espectro de energía se puede dividir en tres rangos: (a) la gran
escala en la cual la energía se inyecta en el rango del sistema (o que
contiene energía), (b) un rango inercial donde el los términos no
lineales dominan sobre los términos disipativos y las cascadas de
energía de escalas grandes a pequeñas, y (c) el rango de pequeña
escala (o disipativo), donde la energía que se está conectando en
cascada desde las escalas más grandes se convierte en calor. Dado que
la energía total (y otros invariantes ideales también) no se modifica
directamente por las interacciones no lineales [p. Frisch, 1995], su
espectro proporciona información importante sobre el estado y la
dinámica del plasma turbulento.

Si el plasma se describe mediante la aproximación MHD de un fluido, su
rango inercial se caracteriza por un espectro de energía que sigue
una ley de potencia $E_k \sim k^{-5/3}$, es decir, el espectro de
Kolmogorov para turbulencia isotrópica, estacionaria e
incompresible. Kolmogorov [1941] predijo esta dependencia de la ley de
potencia para la turbulencia hidrodinámica mediante el análisis
dimensional. Usando mediciones del viento solar a $1$, $2.8$ y
$\si{5}{AU}$, y asumiendo la hipótesis de Taylor [Taylor, 1938],
Matthaeus y Goldstein [1982] obtuvieron espectros de energía
consistentes con el espectro de Kolmogorov.

Sin embargo, las observaciones \emph{in situ} del viento solar han
demostrado que el rango inercial de $k^{-5/3}$ se rompe a cierta
escala, presumiblemente correspondiente a la longitud inercial del ion
[Leamon et al., 2000]. En números de onda mayores que esta ruptura,
los espectros magnéticos exhiben leyes de potencia ligeramente más
pronunciadas [Goldstein et al., 1994, Ghosh et al., 1996, Leamon et
  al., 1998]. Suponiendo estar a escalas más pequeñas que la longitud
inercial del ion, Biskamp et al. [1999] encontró que el espectro de
energía sigue una ley de potencias $k^{-7/3}$. Esta predicción fue
luego confirmada por simulaciones numéricas [Galtier, 2006, Gómez et
  al., 2008]. La explicación tradicional para este régimen de
turbulencia es que está asociado con un proceso de cascada de energía
que involucra ondas dispersivas, tales como los modos ión-ciclotrón
y/o silbato [Ghosh et al., 1996].

Recientemente, Sahraoui et al. [2009] reportó evidencia de un nuevo
punto de ruptura (a escalas aún más pequeñas) en el espectro de
energía magnética de las observaciones del viento solar obtenidas con
la misión Cluster de naves espaciales múltiples. Estos resultados
confirman la ruptura en un número de onda presumiblemente consistente
con la longitud de inercia del ion inverso [Leamon et al., 2000, Smith
  et al., 2001], y encuentran una segunda ruptura en las ondas de onda
más grandes. En el plasma del viento solar a 1 UA, la escala
giroscópica electrónica está muy cerca de la longitud de inercia
electrónica (electron 12 km) y, por lo tanto, no está claro a cuál de
estas escalas corresponde. Los autores confirmaron el espectro de
Kolmogorov en las escalas más grandes, una segunda ley de potencia k
−7/3 en escala intermedia y una ley de potencia más pronunciada k −4.1
en las escalas más pequeñas (más allá del segundo descanso).

Estas rupturas espectrales son probablemente causadas por efectos cinéticos y no pueden describirse mediante la descripción de un fluido MHD. Una posible forma de incluir algunos efectos cinéticos dentro de un marco fluidístico es adoptar un enfoque de múltiples fluidos para reconocer la presencia de varias especies de partículas en un plasma. En particular, estudiamos la turbulencia del viento solar en el marco de una descripción completa de dos fluidos MHD, conservando los efectos de la corriente de Hall, la presión electrónica y la inercia electrónica. El objetivo principal del Capítulo 3 es explorar la física de un modelo completo de dos fluidos y compararlo con mediciones de viento solar in situ.
