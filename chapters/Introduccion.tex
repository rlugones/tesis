\section{Medio interplanetario}
El medio interplanetario se refiere al espacio entre planetas, cometas
y otros objetos del sistema solar. Este espacio no se encuentra vacío,
sino que está lleno de plasma proveniente de la corona solar,
radiación electromagnética, rayos cósmicos, partículas de polvo y
campos magnéticos. El término \emph{viento solar} fue propuesto por
\cite{parker_dynamics_1958} para describir la expansión supersónica
y supermagnetosónica de la corona solar hacia el sistema solar. Esta
expansión se da debido a la diferencia de presiones entre la corona
solar y el espacio interplanetario que la rodea. La diferencia de
presiones vence a la fuerza gravitatoria de la estrella, generando la
expansión del viento solar [\cite{kivelson_introduction_1995}].

La dinámica de los plasmas espaciales, tales como el viento solar,
pueden mostrar grandes diferencias en sus propiedades físicas con
respecto a los plasmas presentes en el laboratorio. Por ejemplo, las
colisiones entre iones o electrones pueden ser bastante frecuentes en
el plasma coronal solar, pero son extremadamente raras en el medio
interplanetario a 1 unidad astronómica ($\SI{1}{AU} =
\SI{1.49e11}{m}$). Dado que algunas de estas condiciones pueden ser
imposibles de reproducir en laboratorios terrestres, el medio
interplanetario ofrece una oportunidad única para investigar una
amplia variedad de procesos plasmáticos. En particular, a
$\SI{1}{AU}$, la densidad numérica media es de $\SI{7}{cm^{-3}}$, la
velocidad de flujo (casi radial) es de $\SI{450}{km/s}$, la
temperatura media es de $\SI{1.2e5}{K}$ y el campo magnético medio
(valor absoluto) es de aproximadamente $\SI{10}{nT}$
[\cite{kivelson_introduction_1995}].

El viento solar varía significativamente de acuerdo a la actividad del
Sol a causa de cambios en el campo magnético solar, el cual termina
contribuyendo al campo magnético interplanetario (IMF) y a la dinámica
general del plasma. Durante su expansión, el viento solar desarrolla
un régimen fuertemente turbulento, que puede estudiarse a través de
mediciones \emph{in situ} [\cite{matthaeus_measurement_1982}]. La
turbulencia aparece como un estado de movimiento muy complejo y
fuertemente irregular en el espacio y el tiempo. Sin embargo, en un
momento dado, un flujo turbulento muestra la presencia de estructuras
organizadas de diferentes tamaños y diferentes tiempos característicos
que interactúan entre sí a medida que el flujo las convecta. Gracias a
las observaciones \emph{in situ} proporcionadas por diferentes naves
espaciales, el flujo del viento solar ofrece la mejor oportunidad para
estudiar directamente la dinámica no lineal en los plasmas espaciales
[\cite{bruno_solar_2005}].




\section{Plasma espacial}

Los plasmas espaciales son gases casi neutros compuestos
principalmente de protones y electrones (es decir, un plasma de
hidrógeno completamente ionizado) que están sujetos a fuerzas
eléctricas y magnéticas. Al mismo tiempo, estas cargas actúan como
fuentes de fuerzas electromagnéticas, haciendo que cada partícula
cargada en el plasma interactúe simultáneamente con una gran cantidad
de otras partículas cargadas, debido a la naturaleza de largo alcance
de estas fuerzas. Este proceso da como resultado un comportamiento
colectivo del plasma.

Existen al menos tres niveles de descripción para modelar la dinámica de
los plasmas espaciales. La principal diferencia entre ellos son las
variables físicas utilizadas para describir el estado del plasma. Cuál
debe elegirse depende del tipo de fenómeno que le interese
estudiar. Los tres enfoques son:
\begin{enumerate}
\item Movimiento de partículas cargadas individuales y su interacción
con el campo eléctrico y magnético.
\item Descripción cinética de una colección de partículas.
\item Descripción fluidística.
\end{enumerate}

La forma más completa de especificar el estado de un plasma es dar las
posiciones y velocidades de todas las partículas y el valor de los
campos en cada punto del espacio. Para un sistema de $N$ partículas,
esta descripción de partículas implica un espacio de fase de $6N$
dimensiones, que se vuelve prohibitivamente grande a medida que
aumenta $N$ (por ejemplo, $N\sim 10^{15}$ para un cubo de
$\SI{1}{km^3}$ en el medio interplanetario). Una posible salida a este
problema es utilizar la teoría cinética. En este enfoque estadístico,
definimos la función de distribución de velocidades $f_s(\vec{x},
\vec{v}, t)$, para cada especie $s$, de manera que $d^3x d^3v
f_s(\vec{x}, \vec{v}, t)$ es la cantidad de partículas en una caja de
tamaño $d^3x$ alrededor de $\vec{x}$, con velocidades en un cubo
$d^3v$ alrededor de $\vec{v}$, para un tiempo $t$. La función de
distribución $f_s$ satisface ecuaciones cinéticas tales como la
ecuación de Vlasov (en el límite no colisional). Estas ecuaciones se
acoplan con las ecuaciones de Maxwell, debido a la autoconsistencia
del campo electromagnético. A pesar de que es posible integrar este
conjunto de ecuaciones numéricamente, poder cubrir un rango realista
de valores de $(\vec{x}, \vec{v}, t)$ se vuelve muy demandante
computacionalmente. Un enfoque más simple comparativamente es
considerar una descripción fluídica del plasma, basada en los momentos
de menor orden de $f_s$ (tales como la densidad de partículas, la
velocidad media del flujo y la presión). Para considerar el hecho de
que las partículas cargadas interactúan con el campo electromagnético,
se acopla la ecuación del fluido con las ecuaciones de Maxwell. Dentro
de este marco, la aproximación más simple es el llamado modelo de
magnetohidrodinámica de un fluido (MHD).

La descripción MHD de un fluido describe adecuadamente la
fenomenología a grandes escalas temporales y espaciales.  Sin embargo,
a escalas espaciales y/o temporales más pequeñas, existen fenómenos
físicos que no pueden reproducirse con la descripción tradicional de
MHD de un fluido. Por ejemplo, para describir adecuadamente el
espectro de energía del viento solar derivado de observaciones
recientes [\cite{sahraoui_evidence_2009}], o para estudiar la reconexión
magnética en el límite sin colisión, se requiere un marco teórico que
se extienda más allá de un fluido MHD. Así es como surgen diversas
extensiones a la teoría de MHD, que permiten analizar muchas situaciones
físicas sin necesidad de pasar a la descripción cinética. Algunas de
esas extensiones a la teoría MHD son Hall-MHD (donde se toma en
consideración el efecto Hall), descripción de dos fluidos (fluido de
electrones por un lado y de iones por otro), electron-MHD (que
describe las escalas más pequeñas del plasma cuando el movimiento de
los electrones es mucho más rápido que el de los iones), etc. A lo
largo de esta tesis, nos ceñiremos a la teoría MHD de un fluido, dado
que esta descripción es suficientemente precisa para los estudios que
realizamos.


\section{Plasma espacial turbulento}

La naturaleza produce habitualmente turbulencia MHD. Se puede
encontrar en varios entornos espaciales, como la corona solar, las
atmósferas planetarias o el medio interplanetario. Como discutimos
anteriormente, el viento solar se expande desde el Sol y penetra las
regiones entre los planetas. Una característica importante para
caracterizar un régimen turbulento estacionario e isotrópico de un
plasma es su espectro de potencia de energía $E_k$, que proporciona la
energía por unidad de número de onda. Al igual que en el caso
paradigmático de la turbulencia hidrodinámica, las interacciones no
lineales en la turbulencia MHD producen un flujo de energía en el
espacio de número de onda que es predominantemente de escalas grandes
a escalas pequeñas. Como veremos con más detalle en la
sección \ref{sec:FundTurbulenciaHD}, el espectro de energía se puede
dividir en tres rangos: (a) la gran escala en la cual la energía se
inyecta en el rango del sistema (\textit{energy containing range}),
(b) un rango inercial donde los términos no lineales dominan sobre los
términos disipativos y las cascadas de energía de escalas grandes a
pequeñas, y (c) el rango de pequeña escala (o disipativa), donde la
energía se disipa en forma de calor. Dado que la energía total (y
otros invariantes ideales también) no se modifica directamente por las
interacciones no lineales [\cite{frisch_turbulence:_1995}], su
espectro proporciona información importante sobre el estado y la
dinámica del plasma turbulento.



\section{Tiempos característicos y correlaciones}

Si bien los tiempos de descorrelación asociados a la transferencia de
energía en turbulencia son los tiempos de descorrelación lagrangianos,
computados siguiendo un elemento material de fluido, la descorrelación
euleriana resulta relevante para realizar predicciones
limitantes. Además, las escalas temporales eulerianas son también
importantes para comprender la dispersión de partículas de prueba
cargadas, como los rayos cósmicos de baja energía
[\cite{bieber_proton_1994}], así como para tener en cuenta la
distribución de las aceleraciones, que está relacionada con la
intermitencia [\cite{nelkin_time_1990}].

Las correlaciones espaciales resultan relevantes en la física de la
turbulencia, ya que la transformada espacial de la correlación a
tiempo de retraso $\tau$ nulo, proveen información acerca de la
distribución espacial de la energía a lo largo de las distintas
escalas. De la misma forma, la correlación temporal a un punto
espacial, variando el tiempo de retraso y transformándolo en
frecuencias, provee información análoga acerca de la distribución
energética a lo largo de las distintas escalas temporales.

A lo largo de la presente Tesis, estudiaremos las correlaciones
eulerianas en tiempo para un dado número de onda o una escala espacial
para el modelo magnetohidrodinámico. Dicho estudio, nos permitirá
tener una mejor comprensión de la turbulencia MHD.

Para ello, primero realizaremos un análisis minucioso de los distintos
tiempos preponderantes el las correlaciones espaciales eulerianas,
para posteriormente enfocarnos en las correlaciones temporales y
espacio-temporales.



\vspace{3cm}

En el \cref{ch:fundamentos} introduciremos ciertos conceptos que nos
permitirán abordar el resto de la Tesis. Comenzaremos con una breve
introducción a turbulencia hidrodinámica, para luego centrarnos en
algunos aspectos fundamentales de turbulencia
magnetohidrodinámica. Nos centraremos particularmente en la
identificación de las escalas temporales relevantes, cómo son
influenciadas por la anisotropía asociada con los campos magnéticos de
gran escala y cómo se alcanza un equilibrio entre las distorsiones no
lineales y las dinámicas asociadas con la propagación de ondas.

En el \cref{ch:P1} desarrollaremos uno de los estudios que realizamos.
Utilizando simulaciones numéricas de turbulencia MHD 3D, analizamos el
comportamiento espacio-temporal de las fluctuaciones del campo
magnético. Consideramos casos con un campo magnético medio de fondo
fuerte, mediano y débil. Computamos la función de correlación temporal
dependiente de la escala espacial (vía el número de onda) para las
distintas simulaciones realizadas, variando el campo magnético
medio. A partir de estas funciones de correlación, computamos el
tiempo de descorrelación y lo comparamos con diferentes tiempos
teóricos pertinentes, con el objeto de estudiar qué fenómeno resulta
preponderante en cada uno de los casos.


En el \cref{ch:P2}, por su parte, expondremos el comportamiento
espacio-temporal de las variables de Els\"asser correspondientes a las
fluctuaciones de los campos magnético y de velocidades, que obtuvimos
mediante simulaciones numéricas de MHD 3D turbulento. Consideramos
casos con campo magnético medio de fondo débil, mediano y fuerte, y
con helicidad cruzada (correlaciones entre el campo de velocidades y
el magnético) nula, pequeña y grande.  Computamos las funciones de
correlación temporal y los tiempos de descorrelación. De esta manera,
pudimos extender el estudio realizado con anterioridad. Además,
observamos contrapropagación de fluctuaciones Alfvénicas debidas a
reflexiones producidas por inhomogeneidades en el campo magnético
total, un resultado inesperado en un principio, pero que corrobora
ciertas predicciones de estudios anteriores. Este efecto se vuelve más
prominente en flujos con alta helicidad cruzada, modificando
fuertemente la propagación de ondas en flujos magnetohidrodinámicos
turbulentos.


Finalmente, en el \cref{ch:conclusiones} expondremos las conclusiones
generales de la presente tesis, así como también la perspectiva de
trabajos futuros.
