Turbulence is an ubiquitous phenomenon in the universe, observed in
scenarios such as neutral flows on Earth, and also in situations like
charged fluids in space. At the same time, plasma represents more than
99\% of visible matter. Despite being so common, plasma turbulence is
extremely complex and complicated to study, since to the difficulties
of a neutral flow (non-linear interaction in the Navier-Stokes
equation, predominantly), the interactions with electromagnetic fields
self-generated and external are added, by coupling between Maxwell and
Navier-Stokes equations.

Among all existing models, magnetohydrodynamic (MHD) turbulence, which
treats plasma as a single fluid, is used in a wide range
of astrophysical and space physics applications. The main reason to
this is that, despite its (relative) simplicity, this model adequately
captures both the macroscopic behavior and the energy cascade from the
large scales to the dissipation scales.

In this thesis we study, using direct 3D numerical simulations, some
of the fundamental aspects of MHD turbulence, such as the spectral
transfer of energy and the multiplicity of temporal scales present,
both in isotropic and anisotropic cases. We analyze the
spatio-temporal behavior of the magnetic fields and velocities by
studying the energy spectrum and the decorrelation times, for cases
with null, small, medium and large mean magnetic field. This allows us
to distinguish the dominant physical effect in a wide variety of
situations.

We also analyze the spatio-temporal spectra of the Els\"sser fields,
independently varying the mean magnetic field (null, small,
intermedite and large) and the cross helicity (null, small and
high). In addition to allowing us to distinguish the dominant effect,
we observe counterpropagation of Alf\'enic fluctuations due to
reflections produced by inhomogeneities in the total magnetic field.
