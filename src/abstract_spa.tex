La turbulencia es un fenómeno ubicuo, observable en escenarios desde
un flujo neutro en la Tierra, hasta en fluidos cargados en el
espacio. A su vez, el plasma representa más del 99\% de la materia
visible. Pese a ser tan común, la turbulencia en plasma es sumamente
compleja y complicada de estudiar, ya que a las dificultados propias
de un flujo neutro (interacción no lineal en la ecuación de
Navier-Stokes, preponderantemente), se agregan las interacciones con
campos electromagnéticos, autogenerados y externos, por el
acoplamiento entre las ecuaciones de Maxwell y de Navier-Stokes.

Entre todos los modelos existentes, la turbulencia
magnetohidrodinámica (MHD), que trata al plasma como un único fluido,
se emplea como modelo para un amplio rango de aplicaciones
astrofísicas y de física espacial. Esto se debe a que, a pesar de su
(relativa) simpleza, captura adecuadamente tanto el comportamiento
macroscópico como la cascada energética desde las grandes escalas
hasta las escalas de disipación.

En esta tesis estudiamos, mediante simulaciones numéricas directas 3D,
algunos de los aspectos fundamentales de la turbulencia MHD, tales
como la transferencia espectral de energía o la multiplicidad de
escalas temporales presentes, tanto en el caso isotrópico como en
casos anisótropos. Analizamos el comportamiento espacio-temporal de
los campos magnético y de velocidades estudiando el espectro de
energías y los tiempos de descorrelación, para los casos de campo
magnético medio nulo, pequeño, mediano y grande. Esto nos permite
distinguir el efecto físico dominante en una amplia variedad de
situaciones.

También analizamos los espectros espacio-temporales de los campos de
Els\"asser, variando independientemente el campo magnético medio
(nulo, pequeño, intermedio y grande) y la helicidad cruzada (nula,
pequeña y grande). Además de permitirnos distinguir el efecto
dominante, observamos contrapropagación de fluctuaciones Alfvénicas
debido a reflexiones producidas por inhomogeneidades en el campo
magnético total.