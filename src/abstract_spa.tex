Dmitruk:
Magnetohydrodynamic 共MHD兲 turbulence has been employed as a physical model for a wide range of
applications in astrophysical and space plasma physics. This Colloquium reviews fundamental aspects
of MHD turbulence, including spectral energy transfer, nonlocality, and anisotropy, each of which is
related to the multiplicity of dynamical time scales that may be present. These basic issues are
discussed based on the concepts of sweeping of the small scales by a large-scale field, which in MHD
occurs due to effects of counterpropagating waves, as well as the local straining processes that occur
due to nonlinear couplings. These considerations give rise to various expected energy spectra, which
are compared to both simulation results and relevant observations from space and astrophysical
plasmas.

Mininni:
This article reviews recent studies of scale interactions in magnetohydro-
dynamic turbulence. The present-day increase of computing power, which
allows for the exploration of different configurations of turbulence in con-
ducting flows, and the development of shell-to-shell transfer functions, has
led to detailed studies of interactions between the velocity and the magnetic
field and between scales. In particular, processes such as induction and dy-
namo action, the damping of velocity fluctuations by the Lorentz force, and
the development of anisotropies can be characterized at different scales. In
this context we consider three different configurations often studied in the
literature: mechanically forced turbulence, freely decaying turbulence, and
turbulence in the presence of a uniform magnetic field. Each configuration is
of interest for different geophysical and astrophysical applications. Local and
nonlocal transfers are discussed for each case. Whereas the transfer between
scales of solely kinetic or solely magnetic energy is local, transfers between
kinetic and magnetic fields are observed to be local or nonlocal depending
on the configuration. Scale interactions in the cascade of magnetic helicity
are also reviewed. Based on the results, the validity of several usual assump-
tions in hydrodynamic turbulence, such as isotropy of the small scales or
universality, is discussed.


P1:
Using direct numerical simulations of three-dimensional magnetohydrodynamic (MHD)
turbulence the spatio-temporal behavior of magnetic field
fluctuations is analyzed.Cases with relatively small, medium and
large values of a mean background magnetic field are considered. The
(wavenumber) scale dependent time correlation function is directly
computed for different simulations, varying the mean magnetic field
value. From this correlation function the time decorrelation is
computed and compared with different theoretical times, namely,
the local non-linear time, the random sweeping time, and the
Alfv\'enic time, the latter being a wave effect. It is observed that
time decorrelations are dominated by sweeping effects, and only at
large values of the mean magnetic field and for wave vectors mainly
aligned with this field time decorrelations are controlled by
Alfv\'enic effects.


P2:
We study the spatio-temporal behavior of the Els\"asser variables
describing magnetic and velocity field fluctuations, using direct
numerical simulations of three-dimensional magnetohydrodynamic
turbulence. We consider cases with relatively small, intermediate,
and large values of a mean background magnetic field, and with null,
small, and high cross-helicity (correlations between the velocity
and the magnetic field). Wavenumber-dependent time correlation
functions are computed for the different simulations. From these
correlation functions, the decorrelation time is computed and
compared with different theoretical characteristic times: the local
non-linear time, the random-sweeping time, and the Alfv\'enic
time. It is found that decorrelation times are dominated by sweeping
effects for low values of the mean magnetic field and for low values
of the cross-helicity, while for large values of the background
field or of the cross-helicity and for wave vectors sufficiently
aligned with the guide field, decorrelation times are controlled by
Alfv\'enic effects. Finally, we observe counter-propagation of
Alfv\'enic fluctuations due to reflections produced by
inhomogeneities in the total magnetic field. This effect becomes
more prominent in flows with large cross-helicity, strongly
modifying the propagation of waves in turbulent magnetohydrodynamic
flows.
