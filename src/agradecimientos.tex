Más allá de mis dificultades para escribir, incluso en abstracto me resulta complicado imaginar cómo se podría plasmar en un papel el sentimiento de gratitud.
No se me ocurre una respuesta satisfactoria a la pregunta ``¿Cómo se verían unas líneas de agradecimientos que reflejen cabalmente lo que quiero significar?''.

La primera tendencia es a pensar que se firman en esta tesis, además de 5 años de trabajo, 5 años de vida.
Pero sería una reflexión cuando menos injusta, cuando en realidad con este trabajo se firma también todo un recorrido anterior incluso a la carrera de grado.
Por eso en lo primero en lo que pienso en estos agradecimientos es en mi familia: en mi viejo Marcelo, mi vieja Silvia y mis hermanas.
Y no puedo evitar recordar cosas que pueden parecer accesorias, como aprender a leer ``Bianchi'' en el costado de un camión yendo (casualmente) a Ciudad Universitaria cuando niño.
Toda mi vida, desde mi infancia, sentí en mi madre y padre un apoyo incondicional hacia la búsqueda de una vocación, que en mi caso fue la de la ciencia natural.
Talvez si tuviera que escarbar y decir qué es lo que más valoro en mis xadres sea la confianza.
En todo momento confiaron ambes en mis decisiones, apuntalando, acompañando, ayudando, levantándome al tropezar y caer mil veces, cayendo elles también conmigo o incluso por su cuenta.
Y ahora que la relación se emparejó (tanto como se puede emparejar la relación entre xadre e hije) tengo además la voluntad inclaudicable de yo también tratar de ayudarles a elles.
Como puedo, a los tropezones, cayendo, a veces empujándoles incluso torpemente.
Pero con una retribución del imperturbable amor que me transmitieron en estos 30 años de vida.

El agradecimiento que más me cuesta traducir en palabras es el que siento por mis hermanas.
Con nadie me peleé, discutí, me enojé tanto como con mis hermanas.
Nadie me peleó, me discutió y a nadie hice enojar tanto como mis hermanas.
(La emoción me eriza un poco [bastante] la piel al escribir esto).
Son tres relaciones muy distintas, todas de una intensidad y un amor incondicional que me sorprende a cada vez.
Con Julieta prácticamente nos criamos juntes.
Fuimos a la misma primaria.
A la misma secundaria.
A la misma facultad.
Y somos tan distintes.
Con Lucía la diferencia de edad es mucho mayor, y entablar una relación más de paridad y menos de ``hermano mayor/hermana menor'' llevó tiempo, costó, llegó y es una de las cosas que más aprecio del mundo.
Y con Sofía la relación aún más distinta.
Más diferencia de edad, nunca convivimos.
Pero espero que sepa, siempre, siempre, que sus hermanes están.
Para todo: lo bueno, lo malo y lo incorregible.

Y afortunadamente la familia no se queda ahí.
Mis abuelas, que siguen aquí y cuyo apoyo es inestimable.
Un abuelo que hace tiempo que se fue, con quien siempre me quedarán las ganas de haber compartido las vivencias más típicas de la adultez.
Y otro abuelo que se fue hace mucho menos.
Si hubiera tenido que elegir una sola persona a quien dedicarle todo este trabajo, sería a él.
Mis tíes y mis primes que están aquí, con cenas y/o almuerzos inagotables para todo tipo de festejo: cumpleaños, años nuevos, navidades, día del padre, etcétera.
A mi Madrina, su compañero y sus hijes (mis ahijades), que viven demasiado lejos, pero siempre están cerca mío.
Y no existe \textit{\textbf{demasiado}} cerca para eso.
La devoción que siento por eses niñes es inexplicable, y el mejor título que tengo es ser su padrino.

A lo largo de los años que pasé en la facultad conocí a muchas personas a las que quiero mucho.
Por cuestiones principalmente relacionadas a factores externos (amigues que se cambian de carrera, que dejan la facultad, que simplemente empezás a dejar de coincidir en las materias) es inevitable que se forme un grupo más cercano.
Eses amigues de la facultad con les que cursaste muchas materias, preparaste finales, hiciste laboratorios, te aventuraste en materias optativas que después dejaste, te dieron una mano para estudiar el día anterior.
Pero luego, con muches de elles también nos acompañamos en esta etapa, la del doctorado, que todes sabemos difícil y que casi inevitablemente llega con tensiones, dudas, desconcierto, desgano y las eventuales ganas de mandar todo al carajo.
Y lo que se podría haber quedado en simplemente una relación eventual de espacio compartido se convierte en una relación entrañable, hermana.
Indeleble.

Rodrigo, un hermano a quien ya llevo más tiempo de mi vida conociéndolo que sin conocerlo.
Quimey, la persona más graciosa que conozco.
Hernán, un tipo tranquilo y con una facilidad sorprendente de volverte al eje.
Leandro, y esas bicicleteadas bizarras como ir a la Mezzeta a las 2AM para comer fugazzeta.
Turiac, que se fue y siempre que vuelve es una fiesta.
Y quiero que sepa que mientras lee esto yo estoy haciendo los trámites para ir a visitarlo.
Compañeros de toda la carrera de grado, con interminables noches en un quincho de precaria protección de los vientos y el frío estudiando para parciales, preparando finales, tomando mates con un agua que a veces hervía.

Santi, el proveedor de anécdotas tan increíbles como esa noche que quedó encerrado en la terraza y estuvo toda la noche arriba.
Vale, la mejor cocinera del mundo, a quién aprovecho para preguntarle cuándo nos juntamos a comer torrejas.
Tati, la Doctora Amor, una hermana de la vida, de citas de Lionel Hutz, cromósforos y ese niñe que ahí viene.
Tute, sus rulos y esa claridad inexplicable para ver el fútbol y la política que siempre me dejan boquiabierto.
Dani, el testigo presencial de mi pulsión por acuchillar puertas.
Algunes que quedamos en Argentina, otres que se fueron.
Pero siempre juntes, y siempre tendermos a nuestros epitafios.

Ese grupo de amigos que se mantiene firme de la secundaria: Gastón, Belo, Pane, Fer.
Todos con distintas vidas, distintos trabajos, distintos horarios.
Más difícil que terminar las cuentas de este doctorado es poder acordar un momento para reunirnos todos.
Pero esas reuniones siempre\ldots fluyen.
Oscilando sana y naturalmente entre distensión y charlas profundas.
Y con una mención especial para Fernando, que estoy seguro de que va a ser el mejor padre del mundo <3.

Al fantástico grupo del WTPC: Rodrigo, Cecilia, Graciela, Pablo.
Con quienes emprendemos un desafío que muchas veces sentimos que nos queda grande pero enfrentamos igual, con vocación, compromiso, dedicación y muchísimo esfuerzo.

Y esas cosas que un poco surgieron últimamente, como les soldades del Dragón: Dani, Pau, Tute, Charly, Nacho.
Ir a ver a Defensores en sus partidos de la B Metropolitana no pudo haber tenido mejor desenlace que esa final del reducido.
1-0 abajo, perdíamos la chance del ascenso, jugando en casa y con un clima de desazón y nerviosismo en todo el estadio.
2do minuto de descuento del segundo tiempo, córner para Defensores.
Tiro de esquina de Quiroga, el Tano Anconetani que va, desarmado pero decidido, al área contraria, cabecea y la empuja el Pájaro Miranda para el empate, el delirio de la gente en el Juan Pasquale y el pase a la definición por penales, en la que Albano Anconetani se termina de vestir de héroe atajando dos penales y Defensores campeón del reducido y ascendiendo al Nacional B y dejando atrás la B Metropolitana.
Y le sobra aguante para no volver.
Gracias a todo el plantel, cuerpo técnico y comisión directiva de Defensores de Belgrano.
Con menciones especiales, además de los ya nombrados, para el Topo Aguirre y Luciano Goux.

Obviamente a todos mis compañeros de oficina: Pasqua, Pedro, Juani, Guille, Fernando.
Los vecinos: Marcelo, Daniel, Andrés, Sebas, Ignacio.
Gracias a todos por el aguante, perdón a todos por ser tan molesto casisiempre.

A todas las composiciones actuales y pretéritas de La Cámpora Exactas.
Y a esos amigos que me dio la política.
Juan Pablo y la decisión política de instalar el hashtag \#BiciAmigos.
Santiago, cuadrazo, verborrágico, inteligentísimo.
Hernán, una persona cuya ternura muchas veces sorprende, confunde, aprende y, sobre todo, enseña.
Pasqua, una mezcla perfecta entre inteligente y humilde, el tipo que mejores preguntas hace.
Maicol, que mientras escribo se está yendo al País Vasco, pero se quedó a ver a Defe campeón del reducido.



A mis compañeres de Villa Pueyrredón, sobre todo a aquelles que me acompañaron en gran parte de esta tesis y fueron fundamentales para darme fuerzas muchas veces: Vicky, Dani, Colo, Nico, Marce.
Y a aquelles con les que seguimos luchando por una política científica y universitaria inclusiva y popular: mis compañeres de Becaries Empoderades y de FEDUBA.
Y vaya un agradecimiento especial en particular para Pablo Perazzi: en un momento en el que estaba perdido (y bastante atacado) terminando la tesis y con todas mis prioridades trastocadas, me lo destrabó diciendo: ``No te preocupes por el resto. Estás en planeta tesis. Se termina en algún momento.''

Y, por supuesto, a la política como concepto.
O, mejor dicho, a cierta política específica.
Cuando hablo de \emph{cierta política específica} me refiero, sin rodeos, a la política instaurada en este país por Néstor Kirchner y Cristina Fernández.
A la política que tomó la decisión de apuntalar una ciencia nacional, un CONICET pujante, con presupuesto, con miras de crecimiento y con un objetivo de poner la Ciencia al servicio del pueblo.
No por una postura vanguardista o por la soberbia que da el conocimiento técnico.
Sino por una vocación política de que la ciencia debe estar al servicio del pueblo, justamente porque es parte del pueblo.
Surge de él y se debe a él.
Porque la ciencia no tiene patria, pero le científique sí la tiene.
Y debe ser uno de los tantos elementos para garantizar la soberanía.
Ojalá así lo vieran todes les dirigentes polítiques del país.

Sin dudas hay dos personas a las que debo resaltar: el primero es mi director, Claudio Dorso.
Me resulta imposible imaginar la dificultad que debe ser dirigir una tesis: no sólo el desafío profesional de partir de une estudiante de grado avanzade y tomar a le licenciade más novate, sino además la responsabilidad de dejar a une estudiante de posgrado avanzade y largar al mundo a le más novate de les doctores\footnote{Si a alguien le parece muy rebuscado esto que estoy diciendo, informo que es algo fundacional que me dijo mi director al terminar de defender mi tesis de grado, hace ya 5 años: ``En esta transición, en este instante, perdiste. Dejás de ser el más avanzado de los estudiantes de grado para ser el más inexperimentado de los estudiantes de posgrado''.}.
No debe de haber sido una tarea sencilla ser mi director a lo largo de esta tesis y debo decir que tampoco se me ocurre cómo podría haber hecho mejor su tarea de director.
Enseñándome todo lo que yo fui capaz de aprender, pero además permitiéndome aprender otras cosas.
Y con una empatía sorprendente para saber cuándo marcar el camino y cuando permitirme rumbear por cosas distintas.
Detrás de esa coraza de hombre rudo al que le gustan los cuchillos hay\ldots un rudo al que le gustan los cuchillos.
No voy a mentir.
Pero también hay una persona comprensiva y profundamente empática con una facilidad que me sorprende (y que agradezco) para manejar esa delgada línea mencionada de marcar el camino y dejar ser.
Y, aunque está de más decirlo, brillante como científico y con una claridad conceptual que creo que jamás voy a poder alcanzar.
Pero que él me enseñó a tratar de alcanzarlo.

Y la otra persona es Charly, una de esas personas sin las cuales no se me ocurriría cómo podría haber transitado este doctorado.
Casi la mitad de mi doctorado lo pasé viviendo con él, en una de las etapas más lindas de mi vida.
Algo que creo que demuestra la profundidad es que buscando las palabras, no se me ocurre más que una copia casi textual de lo que escribí cuando él se doctoró, porque es geuninamente lo que siento.
Compañero de \#EquipoConesa, dos años de Rhapsody, pechugas de pollo, pizzas de dudosa calidad, discusiones con exponentes de la burocracia inmobiliaria, cervezas en la plaza, tereré en los tejados, whisky, mundiales, discusiones de ciencia y reseñas de Intratables y tantos otros recuerdos que no pongo porque temo por el límite de palabras disponibles en las redes sociales.
Un neurótico obsesivo.
Una de las personas con la cabeza más abierta de todas, dispuesto a cambiar de opinión y enseñando en el proceso.
Que todavía dice: ``OK, Rhapsody es medio border. Pero Nightwish tiene calidad\ldots tiene algo''.
Que no sé si es perfectamente consciente de lo que significó su soporte para mí, y por el que le voy a estar agradecido por siempre.
Básicamente un buen tipo, al que quiero un montón.

Entiendo que quizás estos agradecimientos para muches de les que lo leen sean un derroche insoportable de cursilería.
Lamentablemente no soy Saramago para poder escribir aquí un texto sentido, conciso y con la dosis justa de sentimientos que generen empatía sin redundancia.
Y estoy dejando de lado a muchas personas que, por uno u otro motivo, también fueron importantes y fundamentales para esto.
Pero pido que también entiendan: de alguna forma, toda esta tesis, todos estos cinco años, fueron para poder escribir estos agradecimientos.
Y para tratar de aprender, por qué no, mientras los escribo.
Reforzar que este trabajo es producto de un esfuerzo grande por parte mía, pero que habría sido muchísimo más difícil sin todas estas personas a las que agradezco.
Prefiero no interpretar la longitud de estos agradecimientos como lo que evidentemente es (una absoluta falta de síntesis), sino como una demostración de que hay una gran cantidad de factores, que no dependen de la voluntad de une, que me permitieron, bien o mal, como haya salido, llegar a este momento.
Porque nadie se salva sole.
El agradecimiento es inconmensurable, profundo y eterno.