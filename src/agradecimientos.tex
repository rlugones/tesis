Es mucha la gente que me ha acompañado a lo largo de todos estos
años. A riesgo de olvidarme algún nombre, quisiera hacer explícitos
algunos agradecimientos a todas esas personas sin quienes hoy no
estaría donde estoy. Quienes no estén nombrados, tendrán el derecho de
reclamármelo. No obstante, sepan que estoy muy agradecido de todes
aquelles que me han acompañado a lo largo de toda esta etapa.

En primer lugar, quiero agradecerle a Pablo D. No sólo por la
paciencia (que la ha tenido, y mucho), sino también por la
presencia. Ha sido un gran director y una gran persona, entendiendo
mis vueltas y mis vaivenes. También a Pablo M., que me ha dado charlas
y consejos sumamente formativos, tanto académicos como de la vida.

Quiero agradecerle a mis padres, Mariela y Alberto. Mi mamá
lamentablemente no va a poder presenciar este final de etapa, ni las
que están por venir. Sé que le habría encantado verme defender esta
tesis, pero también sé que le habría alcanzado simplemente con verme
feliz. Sin duda alguna, mucho de lo que soy se lo debo a ella y al
derroche inagotable de cariño y amor que me ha dado a lo largo de toda
su vida. Y también quiero agradecerle a mi papá, que por suerte sí
está. Con un amor más rudo, haciendo todo (literalmente todo) para que
pueda tener un gran futuro y una gran vida.

A Anita, mi compañera del alma, una pareja con todo su significado,
que me ha sostenido cada vez que lo he necesitado, que es una de las
causas de mi felicidad diaria, y que ha sido uno de los grandes
motores de cambios en mi vida.

A mis amigos más íntimos. Pablo, un hermano que conozco hace ya !`19!
años y con quien me he embarcado en una infinidad de proyectos, todos
exitosos (de una manera u otra).  A Quimey, una de las personas más
graciosas que existen (una vez que entra en confianza).  A Leandro,
que está presente cada vez que lo necesito, en las buenas y en las
malas.  A Sebastián, con quien vivo charlando de proyectos presentes y
futuros.  A Francisco (alias Pipo), con quien me junto mucho menos de
lo que querría, pero que en cada encuentro parece que el tiempo no
hubiese pasado.  A Hernán y a Turiax, con quienes la distancia nos ha
distanciado un poco, pero a quienes quiero muchísimo y siempre es una
alegría enorme verlos.

A Ignacio y Germán, mis hermanos de sangre, que me han apoyado y
ayudado mucho en estos últimos muy difíciles meses. Y a Sofi, Delfi y
Benja, mis sobrines que adoro aunque les vea poquito.

Al grupo del WTPC: Pablo (otra vez), Cecilia y Graciela. Ceci, una de
las personas más dulces que conozco; y Gra, una mujer que admiro
profundamente.

A Ninja, principalmente a Pablo (una vez más) y a Hernán, pero sin
olvidarme de Nico, Mariel, Juanpa, Paula, Fer. Sin elles, este
hermoso proyecto (ya real) que es Ninja no podría existir.

A mis suegros, Fito y Cristina, y también a Juampi, Tere, Ñata, Norma y Jorge,
quienes desde un primer momento me recibieron en su familia como si
fuera uno más.

A Jesi y a Sarita, que han sido unas grandes vecinas y amigas.

A Marie y a Mati, gestores de la OMF, con quienes hemos compartido
mucho stress de organización de último momento y muchas cervezas
posteriores a cada olimpiada exitosa.

A Nadia y a Pedro, dos amigos que espero que trasciendan mi paso por
la facu.

A Pablo Piteo (no el mismo Pablo que antes), siempre presente.

A mis compañeros de fútbol en los distintos equipos en los que he
estado (interfacultades, Que Entre Toda, Africanos y La Milagrosa),
que me dan la oportunidad de disfrutar de este hermoso deporte. En
especial al Negro, que ha conseguido que sea capaz de dar un pase al
pie.

A Mariano, por todas sus ayudas con los infinitos papeles.

A Yuditsabet y al Chino, que han sido magníficos compañeres de
oficina.

Por último, pero no por eso menos importante, a todo el grupo de FLiP,
principalmente a Carlitos, a Mauro y a Nico.

A todes ustedes, les dedico esta tesis.
